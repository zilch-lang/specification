% See: https://saswat.padhi.me/blog/2019-09_variadic-macros-in-latex/index.html
%%%%
% View a sample document: https://www.overleaf.com/read/bwkjcwcktsqd
%
% USAGE :: \VARIADIC {name} {before} {start} {mid} {stop} {after}
\newcommand{\VARIADIC}[6]{%
	\expandafter\newcommand\csname Gobble#1Arg\endcsname[2]{%
		\csname Check#1Arg\endcsname{##1#4##2}%
	}%
	\expandafter\newcommand\csname Check#1Arg\endcsname[1]{%
		\csname @ifnextchar\endcsname\bgroup{\csname Gobble#1Arg\endcsname{##1}}{#2{##1#5}#6}%
	}%
	\expandafter\newcommand\csname #1\endcsname[1]{%
		\csname Check#1Arg\endcsname{#3##1}%
	}%
}

\namespace{nstar:rules}{%
\newcommand*\CodeTypeSequent[7]{\ensuremath{{#1};{#2};{#3};{#4};{#5}\vdash^\text{\tiny T}_\text{code}{#6:#7}}}
\newcommand*\DataTypeSequent[3]{\ensuremath{{#1}\vdash^\text{\tiny T}_\text{data}{#2:#3}}}
\newcommand*\KindSequent[3]{\ensuremath{{#1}\vdash^\text{\tiny K}{#2%
			\ifx&#3&%
			\else%
			:#3%
			\fi}}}
\newcommand*\ISequent[9]{\ensuremath{{#1};{#2};{#3};{#4};{#5}\vdash^\text{\tiny I}{#6}\dashv{#7};{#8};{#9}}}
}{}

\namespace{nstar:grammar}{%
%% TYPES
\def\Tformat#1{\color{OliveGreen}\relax\ifmmode\bm{#1}\else\textbf{#1}\fi}
%
\newcommand*\Tunsigned[1]{\ensuremath{\text{\Tformat{u#1}}}}
\newcommand*\Tsigned[1]{\ensuremath{\text{\Tformat{s#1}}}}
\VARIADIC{Tstruct}{\ensuremath}{(}{,\,}{)}{}
\VARIADIC{Tunion}{\ensuremath}{\langle}{,\,}{\rangle}{}
\newcommand*\Tbang{\ensuremath{\text{\Tformat !}}}
\newcommand*\Tptr[1]{\ensuremath{{\Tformat{{}^\ast}}#1}}
\newcommand*\Tchar{\ensuremath{\text{\Tformat{char}}}}
\newcommand*\Tcontext[3]{\ensuremath{\{ #1 \mid #2 \to #3 \}}}
\newcommand*\Tforall[2]{\ensuremath{{\Tformat \forall}(#1){\Tformat .}\,{#2}}}
\newcommand*\Tstack[2]{\ensuremath{#1\,{\Tformat \dblcolon}\,#2}}
\VARIADIC{Tsstack}{\ensuremath}{}{\,{\Tformat \dblcolon}\,}{}{}
\newcommand*\Tkind[1]{\ensuremath{\text{\Tformat{T#1}}}}

\def\Ts{\T{s}}
\def\Ta{\T{a}}
\def\Tc{\T{c}}
\let\T\Tkind

%% EXPRESSIONS
\newcommand*\Echar[1]{\ensuremath{\text{'#1'}}}
\newcommand*\EInt[2][]{\ensuremath{#1#2}}
\newcommand*\Eregister[1]{\ensuremath{\text{\bfseries \%r#1}}}
\newcommand*\Ebyteoff[2]{\ensuremath{#1(#2)}}
\newcommand*\Ebaseoff[2]{\ensuremath{#1 [#2]}}
\VARIADIC{Earray}{\ensuremath}{[}{,\,}{]}{}
\newcommand*\Estring[1]{\ensuremath{\text{"#1"}}}
\VARIADIC{Estruct}{\ensuremath}{(}{,\,}{)}{}

\let\r\Eregister

%% INSTRUCTIONS
\definecolor{kw}{HTML}{5B269A}
\def\Iformat{\ttfamily\color{kw}}
%
\makeatletter
\newcommand*\definstr[2][0]{%
	\ifnum#1=0%
		\expandafter\newcommand\csname I#2\endcsname{\ensuremath{\text{\Iformat #2}}}
	\else%
		\expandafter\VARIADIC{I#2}{\ensuremath}{\text{\Iformat #2}\ }{,\,}{}{}
	\fi%
}
\makeatother
%
\definstr{nop}
\definstr[1]{jmp}
\definstr[1]{call}
\definstr{ret}
\definstr{halt}
\definstr[2]{sld}
\definstr[2]{sst}
\definstr[1]{salloc}
\definstr{sfree}
\definstr[1]{sref}
\definstr[2]{ld}
\definstr[2]{st}
\definstr[2]{mv}
}{}


\let\oldpart\part
\renewcommand{\partmark}{}
\renewcommand{\part}[1]{\oldpart{#1}\renewcommand{\partmark}{#1}}


\pagestyle{fancy}
\fancyhf{}
\renewcommand{\chaptermark}[1]{\markboth{#1}{#1}}
\fancyhead[L]{\textsc{\partmark}}
\fancyhead[R]{\textsc{\leftmark}}
\makeatletter
\fancyfoot[L]{\textsc{\@title}}
\makeatother
\fancyfoot[R]{\textsc{Page \thepage} of \pageref*{LastPage}}
\renewcommand{\headrulewidth}{0.4pt}
\renewcommand{\footrulewidth}{0.4pt}

\patchcmd{\chapter}{\thispagestyle{plain}}{\thispagestyle{fancy}}{}{}
\patchcmd{\oldpart}{\thispagestyle{plain}}{\thispagestyle{empty}}{}{}


\setlength{\cftpartindent}{0em}
\setlength{\cftchapindent}{2em}
\setlength{\cftsecindent}{4em}
\setlength{\cftsubsecindent}{6em}
\setlength{\cftsubsubsecindent}{8em}



\renewcommand\epigraphflush{flushright}
\renewcommand\epigraphsize{\normalsize}
\setlength\epigraphwidth{0.7\textwidth}


\makeatletter

\@addtoreset{chapter}{part}
\makeatother

\definecolor{mintedbg}{rgb}{.96,.96,.96}

\setminted{
	breaklines=true,
	mathescape=true,
	breaksymbolleft={},
	autogobble=true,
	xleftmargin=\parindent,
	xrightmargin=\parindent,
	frame=single,
	framerule=\fboxrule,
	framesep=1.5\fboxsep,
	style=xcode,
}
\makesavenoteenv{minted}

\BeforeBeginEnvironment{minted}{\begin{minipage}{\textwidth}}
		\AfterEndEnvironment{minted}{\end{minipage}}

\def\nstarlexer{lexer/nstar.py:NStarLexer -x}
\def\zilchlexer{lexer/zilch.py:ZilchLexer -x}

\setcounter{tocdepth}{4}
\setcounter{secnumdepth}{4}

\defglsentryfmt{\glsgenentryfmt
	\ifglsused{\glslabel}{}{$^*$}}

\renewcommand{\glossarysection}[2][\theglstoctitle]{%
	\def\theglstoctitle{#2}%
}

\makeglossaries{}
\input{glossary}

\hypersetup{
	colorlinks=false
}

\graphicspath{{./res/}}

\titleformat{\chapter}[block]{\normalfont\LARGE\bfseries}{\thechapter.}{1em}{}
\titlespacing*{\chapter}{0pt}{-19pt}{19pt}

\definecolor{titlepagecolor}{cmyk}{0,.62,.62,.09}

\addbibresource{main.bib}


\newcolumntype{P}[1]{>{\centering\arraybackslash}p{#1}}
\newcolumntype{Y}[0]{>{\centering\arraybackslash}X}


\usetikzlibrary{arrows, shapes}
\tcbuselibrary{skins, breakable}


\def\nstar{N$^{\star}$}

\setlength{\fboxsep}{2ex}

\hypersetup{
	colorlinks=true,
	linkcolor=black,
	filecolor=black,
	anchorcolor=black,
	citecolor=RedViolet,
	urlcolor=RoyalPurple,
	pdftitle={The magical Zilch book},
	bookmarks=true,
}

% Params:
% - #1: color
% - #2: left icon
\newtcolorbox[]{alertmessage}[3][]{%
	enhanced, breakable,
	colback=#2!20,
	colframe=#2!80,
	coltitle=#2!80,
	boxrule=1pt,
	arc=5pt,
	detach title,
	%attach title to upper={\tcblower},
	%sidebyside,
	fonttitle={\LARGE},
	title={#3},
	%center upper,
	%lefthand width=0.75cm,
	%sidebyside gap=6mm,
	%lower separated=false,
	left=1.25cm,
	overlay={
			\node[minimum width=0.75cm, anchor=south, xshift=0.7cm, yshift=-0.45cm] at (frame.west) {\tcbtitle};
		},
	#1
}

\newenvironment{warningbox}[1][]{\begin{alertmessage}[#1]{Orange}{\faicon{exclamation-circle}}}{\end{alertmessage}}
\newenvironment{infobox}[1][]{\begin{alertmessage}[#1]{Blue}{\faicon{info-circle}}}{\end{alertmessage}}
\newenvironment{commentbox}[1][]{\begin{alertmessage}[#1]{OliveGreen}{\faicon{comment}}}{\end{alertmessage}}
\newenvironment{checkbox}[1][]{\begin{alertmessage}[#1]{Green}{\faicon{check-circle}}}{\end{alertmessage}}
\newenvironment{errorbox}[1][]{\begin{alertmessage}[#1]{Red}{\faicon{times-circle}}}{\end{alertmessage}}

%%%%%%%%%%%%%%%%%%%%%%%%%%%%%%%%%%%%%%%%%%%%%%%%%%%%%%%%%%%%%%%%%%%%%%%%%%%%%%%%%%%%%%%%%%%%%%%%%%%%%%%%

\newcommand{\iseq}{\ensuremath{\overset{\scriptscriptstyle ?}{=}}}
